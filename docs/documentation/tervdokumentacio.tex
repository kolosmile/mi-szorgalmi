\documentclass[12pt,a4paper]{article}
\usepackage[utf8]{inputenc}
\usepackage[T1]{fontenc}
% \usepackage[hungarian]{babel} % Disabled due to compilation errors
\usepackage{geometry}
\usepackage{graphicx}
\usepackage{hyperref}
% \usepackage{listings}
% \usepackage{xcolor}

\geometry{
 left=25mm,
 top=25mm,
 right=25mm,
 bottom=25mm
}

\hypersetup{
    colorlinks=true,
    linkcolor=black,
    filecolor=magenta,      
    urlcolor=blue,
}

\title{Kör illesztése adott ponthalmazra 2.}
\author{Mile Kolos}
\date{2025. november 18.}

\begin{document}

% --- Címlap ---
\begin{titlepage}
    \centering
    \vspace*{1cm}
    
    \Huge
    \textbf{Mesterséges Intelligencia Szorgalmi Feladat}
    
    \vspace{0.5cm}
    \LARGE
    Téma: Kör illesztése adott ponthalmazra 2.
    
    \vspace{1.5cm}
    
    \textbf{Készítette:} Mile Kolos \\
    \textbf{Neptun-kód:} OXEZ80 \\
    \textbf{E-mail:} kolosk5@gmail.com
    
    \vspace{0.5cm}
    
    \textbf{Konzulens:} Dr. Póka György
    
    \vfill
    
    \Large
    Budapesti Műszaki és Gazdaságtudományi Egyetem\\
    Gépészmérnöki Kar\\
    Gyártástudomány és Technológia Tanszék
    
    \vspace{0.8cm}
    
    Budapest, 2025. november 18.
    
\end{titlepage}

% --- Tartalomjegyzék ---
\tableofcontents
\newpage

% --- Feladat leírása ---
\section{A feladat leírása}

\subsection{A probléma bemutatása}

A feladat egy klasszikus geometriai-optimalizálási probléma modern, mesterséges intelligencia alapú megközelítése. A cél egy adott 2D ponthalmazra a \textbf{legkisebb befoglaló kör} (minimum enclosing circle) meghatározása. A kihívást az jelenti, hogy a ponthalmaz nem ideális, hanem valós mérési vagy digitalizálási folyamatokból származó hibákat szimulál, úgymint:

\begin{itemize}
    \item \textbf{Alakhiba:} A pontok nem egy tökéletes köríven helyezkednek el.
    \item \textbf{Véletlen zaj:} Minden pont pozíciója egy kis mértékű, véletlenszerű eltolással terhelt.
    \item \textbf{Kiugró pontok (outlierek):} A ponthalmaz tartalmaz néhány, a fő csoporttól távol eső, hibás mérési eredményt szimuláló pontot.
\end{itemize}

Az algoritmusnak robusztusnak kell lennie, hogy ezekkel a hibákkal megbirkózzon, és a definíció szerint megtalálja azt a legkisebb sugarú kört, amely a ponthalmaz \textit{összes} elemét tartalmazza.

\subsection{Célkitűzések}

A projekt során a következő fő célokat kellett elérni:

\begin{enumerate}
    \item \textbf{Ponthalmaz generálása:} Egy olyan programmodul létrehozása, amely képes paraméterezhető módon, a fent említett hibákkal terhelt ponthalmazokat generálni.
    \item \textbf{Algoritmus fejlesztése:} Egy mesterséges intelligencia alapú algoritmus (esetünkben genetikus algoritmus) implementálása, amely a generált ponthalmazra illeszti a legkisebb befoglaló kört.
    \item \textbf{Kiértékelés:} A kifejlesztett módszer teljesítményének objektív elemzése. Vizsgálni kell a futási időt a bemeneti adatok méretének függvényében, elemezni kell az algoritmus paramétereinek (pl. mutációs ráta) hatását az eredményre, és vizsgálni kell a megoldás konvergenciáját.
    \item \textbf{Dokumentálás:} A teljes folyamat, a módszer és az elért eredmények részletes dokumentálása a követelményeknek megfelelően.
\end{enumerate}

\newpage

% --- Megoldás elve ---
\section{A megoldás elve, módszere}

A probléma megoldására egy Python-alapú szoftveres megoldás készült, amely egy genetikus algoritmust alkalmaz a legkisebb befoglaló kör megkeresésére.

\subsection{Technológiai háttér}

A projekt az alábbi technológiákra épül:
\begin{itemize}
    \item \textbf{Nyelv:} Python 3
    \item \textbf{Könyvtárak:}
    \begin{itemize}
        \item \texttt{NumPy}: A numerikus számítások (távolságmérés, koordináta-manipuláció) hatékony elvégzéséért felel. Nélkülözhetetlen a nagy mennyiségű pontadat gyors feldolgozásához.
        \item \texttt{Matplotlib}: Az adatok és eredmények vizualizációjáért felel. Segítségével ábrázoljuk a generált ponthalmazokat, az illesztett köröket és a kiértékelés során kapott grafikonokat.
    \end{itemize}
\end{itemize}

\subsection{Ponthalmaz generálása hibákkal}

A kiindulási adathalmazt egy dedikált modul (\texttt{point\_generator.py}) hozza létre, amely egy ideális körből indul ki, és szisztematikusan hibákat ad hozzá:
\begin{enumerate}
    \item \textbf{Alap kör definiálása:} Egy $(x, y)$ középpontú, $r$ sugarú körön egyenletesen elhelyezünk $N$ számú pontot.
    \item \textbf{Alakhiba hozzáadása:} A pontok koordinátáit enyhén torzítjuk, ami egy ellipszis-szerű alakot eredményez.
    \item \textbf{Véletlen zaj hozzáadása:} Minden pont $x$ és $y$ koordinátájához egy normális eloszlású véletlen értéket adunk, ami a mérési pontatlanságot szimulálja.
    \item \textbf{Kiugró pontok generálása:} A fő ponthalmazon kívül, nagyobb távolságra elhelyezünk néhány pontot, amelyek a durva mérési hibákat reprezentálják.
\end{enumerate}

\newpage

% --- Implementáció bemutatása ---
\section{Az implementáció bemutatása}

A szoftver Python 3 nyelven készült, moduláris felépítéssel, hogy a különböző funkciók (pontgenerálás, algoritmus, kiértékelés) logikailag elkülönüljenek.

\subsection{A projekt struktúrája}

A projekt fő mappái és fájljai a következők:

\begin{verbatim}
.
|-- docs/                 # Dokumentacio es feladatkiiras
|-- src/                  # A Python forraskodok
|   |-- point_generator.py
|   |-- genetic_algorithm.py
|   |-- evaluation.py
|   |-- main.py
|-- venv/                 # Virtualis kornyezet
|-- .gitignore
|-- requirements.txt      # Projekt fuggosegek
|-- README.md
\end{verbatim}

\end{document}
