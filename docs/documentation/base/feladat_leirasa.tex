\documentclass{article}
\usepackage[utf8]{inputenc}
\usepackage[T1]{fontenc}
% \usepackage[hungarian]{babel}
\usepackage{geometry}

\geometry{
 a4paper,
 total={170mm,257mm},
 left=20mm,
 top=20mm,
}

\begin{document}

\section*{3. A feladat leírása}

\subsection*{A probléma bemutatása}

A feladat egy klasszikus geometriai-optimalizálási probléma modern, mesterséges intelligencia alapú megközelítése. A cél egy adott 2D ponthalmazra a \textbf{legkisebb befoglaló kör} (minimum enclosing circle) meghatározása. A kihívást az jelenti, hogy a ponthalmaz nem ideális, hanem valós mérési vagy digitalizálási folyamatokból származó hibákat szimulál, úgymint:

\begin{itemize}
    \item \textbf{Alakhiba:} A pontok nem egy tökéletes köríven helyezkednek el.
    \item \textbf{Véletlen zaj:} Minden pont pozíciója egy kis mértékű, véletlenszerű eltolással terhelt.
    \item \textbf{Kiugró pontok (outlierek):} A ponthalmaz tartalmaz néhány, a fő csoporttól távol eső, hibás mérési eredményt szimuláló pontot.
\end{itemize}

Az algoritmusnak robusztusnak kell lennie, hogy ezekkel a hibákkal megbirkózzon, és a definíció szerint megtalálja azt a legkisebb sugarú kört, amely a ponthalmaz \textit{összes} elemét tartalmazza.

\subsection*{Célkitűzések}

A projekt során a következő fő célokat kellett elérni:

\begin{enumerate}
    \item \textbf{Ponthalmaz generálása:} Egy olyan programmodul létrehozása, amely képes paraméterezhető módon, a fent említett hibákkal terhelt ponthalmazokat generálni.
    \item \textbf{Algoritmus fejlesztése:} Egy mesterséges intelligencia alapú algoritmus (esetünkben genetikus algoritmus) implementálása, amely a generált ponthalmazra illeszti a legkisebb befoglaló kört.
    \item \textbf{Kiértékelés:} A kifejlesztett módszer teljesítményének objektív elemzése. Vizsgálni kell a futási időt a bemeneti adatok méretének függvényében, elemezni kell az algoritmus paramétereinek (pl. mutációs ráta) hatását az eredményre, és vizsgálni kell a megoldás konvergenciáját.
    \item \textbf{Dokumentálás:} A teljes folyamat, a módszer és az elért eredmények részletes dokumentálása a követelményeknek megfelelően.
\end{enumerate}

\end{document}
